\begin{thebibliography}{99}
\setlength{\parskip}{0.5\baselineskip}

\bibitem{waterman}
    Waterman A., 
    \textit{Design of the RISC-V Instruction Set Architecture},
    PhD diss.,
    Electrical Engineering and Computer Sciences,
    University of California at Berkeley,
    2016,
    UCB/EECS-2016-1.

\bibitem{reader}
    Patterson D., Waterman A., 
    \textit{The RISC-V Reader: An Open Architecture Atlas},
    First edition,
    Strawberry Canyon,
    2017.

\bibitem{hennessy17}
    Hennessy J., Patterson D., 
    \textit{Computer Architecture: A Quantitative Approach},
    Sixth edition,
    Morgan Kaufmann,
    2017.

\bibitem{scoreboard}
    Thornton J.,
    ``Parallel operation in the control data 6600'',  
    \textit{Proceedings of the fall joint computer conference, part II: very high speed computer systems},
    vol.~26, pp.~33--40, 
    1965.

\bibitem{mittal19}
    Mittal S.,
    ``A Survey of Techniques for Dynamic Branch Prediction'',  
    \textit{Concurrency and Computation: Practice and Experience},
    vol.~31, no.~1, 
    2019.

\bibitem{smith98}
    Smith J.,
    ``A study of branch prediction strategies'',  
    \textit{25 Years of the International Symposia on Computer Architecture},
    pp.~202--215, 
    1998.

\bibitem{gross82}
    Gross T., Hennessy J.,
    ``Optimizing delayed branches'',  
    \textit{ACM SIGMICRO Newsletter},
    vol.~13, pp.~114--120, 
    1982.
    
\bibitem{yeh91}
    Yeh T., Patt Y.,
    ``Two-level adaptive training branch prediction'',  
    \textit{Proceedings of the 24th Annual International Symposium on Microarchitecture},
    pp.~51--61, 
    1991.

\bibitem{yeh93}
    Yeh T., Patt Y.,
    ``A comparison of dynamic branch predictors that use two levels of branch history'',  
    \textit{Proceedings of the 20th Annual International
    Symposium on Computer Architecture},
    pp.~257--266, 
    1993.

\bibitem{seznec06}
    Seznec A., Michaud P.,
    ``A case for (partially)-tagged geometric history length predictors'',  
    \textit{Journal of Instruction Level Parallelism},
    vol.~8, pp.~1--23, 
    2006.

\bibitem{jimenez01}
    Jimenez D., Lin C.,
    ``Dynamic branch prediction with perceptrons'',  
    \textit{Proceedings HPCA Seventh International Symposium on High-Performance Computer Architecture},
    pp.~197--206, 
    2001.

\bibitem{axi}
    ARM, 
    \textit{AMBA AXI and ACE Protocol Specification},
    2017.

\bibitem{resets}
    Cummings C.E., Mills D.,
    ``Synchronous Resets? Asynchronous Resets? I am so confused! How will I ever know which to use?'',  
    \textit{Synopsys Users Group Conference, San Jose, CA, 2002},
    User Papers, 
    2002.

\bibitem{mcfarling93}
    McFarling S.,
    ``Combining branch predictors'',  
    \textit{Digital Western Research Laboratory},
    vol.~49, technical report TN-36, 
    1993.

\bibitem{lee84}
    Lee J.K.F., Smith A.J.,
    ``Branch prediction strategies and branch target buffer design'',  
    \textit{Computer},
    vol.~1, pp.~6--22, 
    1984.

\bibitem{perleberg93}
    Perleberg C.H., Smith A.J.,
    ``Branch target buffer design and optimization'',  
    \textit{IEEE transactions on computers},
    vol.~42(4), pp.~396--412, 
    1993.

\bibitem{dc}
    Synopsys, 
    \textit{Design Compiler User Guide},
    2011.

\bibitem{parasanna17}
    Parasanna S., Sarma R., Balasubramanian S.,
    ``A study on improving branch prediction accuracy in the context of conditional branches'',  
    \textit{Int J Eng Technol Sci Res},
    vol.~4, pp.~654--662, 
    2017.

\bibitem{michaud18}
    Michaud P.,
    ``An Alternative TAGE-like Conditional Branch Predictor'',  
    \textit{ACM Transactions on Architecture and Code Optimization (TACO)},
    vol.~15.3, p.~30, 
    2018.

\end{thebibliography}