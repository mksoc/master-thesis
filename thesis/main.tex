\documentclass[%
corpo=12pt,
tipotesi=magistrale,
evenboxes,
%libro,
twoside
]{toptesi}

%%%%%%%%%%%%%%%%%%%%%%%%%%%%%%%%%%%%%%%%%%%%%%%%%%%%%%%%%%
% Impostazioni per comporre con pdfLaTeX
\usepackage[utf8]{inputenc}
\usepackage[T1]{fontenc}
%\usepackage{newtxtext}% o altro font
%%%%%%%%%%%%%%%%%%%%%%%%%%%%%%%%%%%%%%%%%%%%%%%%%%%%%%%%%%

% Link
\usepackage{hyperref}
\hypersetup{%
    pdfpagemode={UseOutlines},
    bookmarksopen,
    pdfstartview={FitH},
    colorlinks,
    linkcolor={blue},
    citecolor={blue},
    urlcolor={blue}
  }

\usepackage{lipsum}

\begin{document}
\english

%%%%%%%%%%%%%%%% Frontespizio %%%%%%%%%%%%%%%%
% Frontespizio in inglese
\CorsoDiLaureaIn{Master's Degree in}
\TesiDiLaurea{Master Thesis}
\CandidateName{Candidate}
\AdvisorName{Supervisor}

\begin{frontespizio}
    %\ateneo{Politecnico di Torino}
    \logosede[30mm]{img/polito_logo.png}
    \corsodilaurea{Electronic Engineering}
    \titolo{Ciao RISC-V}
    \sottotitolo{Chissà se mai finiremo}
    \relatore{Prof. Maurizio \textsc{Martina}}
    \sedutadilaurea{Academic year 2018-2019}
    \candidato{Marco \textsc{Andorno}}    
\end{frontespizio}
%%%%%%%%%%%%%%%%%%%%%%%%%%%%%%%%%%%%%%%%%%%%%%

\sommario
\lipsum[1-1]

\ringraziamenti
\lipsum[2-2]

\figurespagetrue
\tablespagetrue
\indici

\chapter{Introduction}
\lipsum[1-4]

\begin{thebibliography}{99}
\setlength{\parskip}{0.5\baselineskip}

\bibitem{waterman}
    Waterman A., 
    \textit{Design of the RISC-V Instruction Set Architecture},
    PhD diss.,
    Electrical Engineering and Computer Sciences,
    University of California at Berkeley,
    2016,
    UCB/EECS-2016-1.

\bibitem{reader}
    Patterson D., Waterman A., 
    \textit{The RISC-V Reader: An Open Architecture Atlas},
    First edition,
    Strawberry Canyon,
    2017.

\bibitem{hennessy17}
    Hennessy J., Patterson D., 
    \textit{Computer Architecture: A Quantitative Approach},
    Sixth edition,
    Morgan Kaufmann,
    2017.

\bibitem{scoreboard}
    Thornton J.,
    ``Parallel operation in the control data 6600'',  
    \textit{Proceedings of the October 27–29, 1964, fall joint computer conference, part II: very high speed computer systems},
    vol. 26, pp. 33-40, 
    1965.

\bibitem{mittal19}
    Mittal S.,
    ``A Survey of Techniques for Dynamic Branch Prediction'',  
    \textit{Concurrency and Computation: Practice and Experience},
    vol. 31, no. 1, 
    2019.
    
\end{thebibliography}

\end{document}