\documentclass[%
corpo=12pt,
tipotesi=magistrale,
%libro,
twoside
]{toptesi}

% Frontespizio in inglese
\CorsoDiLaureaIn{Master's Degree in}
\TesiDiLaurea{Master Thesis}
\CandidateName{Candidate}
\AdvisorName{Supervisor}

%%%%%%%%%%%%%%%%%%%%%%%%%%%%%%%%%%%%%%%%%%%%%%%%%%%%%%%%%%
% Impostazioni per comporre con pdfLaTeX
\usepackage[utf8]{inputenc}
\usepackage[T1]{fontenc}
%\usepackage{newtxtext}% o altro font
%%%%%%%%%%%%%%%%%%%%%%%%%%%%%%%%%%%%%%%%%%%%%%%%%%%%%%%%%%

% Link
\usepackage{hyperref}
\hypersetup{%
    pdfpagemode={UseOutlines},
    bookmarksopen,
    pdfstartview={FitH},
    colorlinks,
    linkcolor={blue},
    citecolor={blue},
    urlcolor={blue}
  }

\usepackage{lipsum}

\begin{document}
\english

\begin{frontespizio*}
    \ateneo{Politecnico di Torino}
    \logosede{img/polito_logo.png}
    \corsodilaurea{Electronic Engineering}
    \titolo{Ciao RISC-V}
    \sottotitolo{Chissà se mai finiremo}
    \relatore{Prof. Maurizio \textsc{Martina}}
    \sedutadilaurea{Academic year 2018-2019}
    \candidato{Marco \textsc{Andorno}}    
\end{frontespizio*}

\sommario
\lipsum[1-1]

\ringraziamenti
\lipsum[2-2]

\indici

\chapter{Introduction}
\lipsum[1-4]

\end{document}